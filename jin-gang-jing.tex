\documentclass[a6paper, 22pt, twocolumn]{cvertbook}
\usepackage{hyperref}
\hypersetup{
    pdftitle={般若菠萝密金刚经}
    pdfauthor={xue35}
}

\setCJKmainfont{中华书局宋体00平面}
\setCJKfallbackfamilyfont{\CJKrmdefault}
{{中华书局宋体02平面},{中华书局宋体15平面}}

\judou      %启用句读模式,默认为不启用
\setlength{\vertbookoffset}{0.4cm}    %设置装订线距离,默认为1cm

\begin{document}
\inmain


\chapter{第壹品-法會因由分}
如是我聞。壹時佛在舍衛國。祗樹給孤獨園。與大比丘眾。千二百五十人俱。爾時世尊。食時。著衣持缽。入舍衛大城乞食。於其城中。次第乞已。還至本處。飯食訖。收衣缽。洗足已。敷座而坐。
\chapter{第二品-善現啟請分}
時長老須菩提。在大眾中。即從座起。偏袒右肩。右膝著地。合掌恭敬。而白佛言。希有世尊。如來善護念諸菩薩。善付囑諸菩薩。世尊。善男子。善女人。發阿耨多羅三藐三菩提心。應雲何住,雲何降伏其心。佛言。善哉善哉。須菩提。如汝所說。如來善護念諸菩薩。善付囑諸菩薩。汝今諦聽。當為汝說。善男子。善女人。發阿耨多羅三藐三菩提心。應如是住,如是降伏其心。唯然。世尊。願樂欲聞。
\chapter{第三品-大乘正宗分}
佛告須菩提。諸菩薩摩訶薩。應如是降伏其心。所有壹切眾生之類。若卵生。若胎生。若濕生。若化生。若有色。若無色。若有想。若無想。若非有想。非無想。我皆令入無余涅盤而滅度之。如是滅度無量無數無邊眾生。實無眾生得滅度者。何以故。須菩提。若菩薩有我相。人相。眾生相。壽者相。即非菩薩。
\chapter{第四品-妙行無住分}
復次。須菩提。菩薩於法。應無所住行於布施。所謂不住色布施。不住聲香味觸法布施。須菩提!菩薩應如是布施。不住於相。何以故?若菩薩不住相布施。其福德不可思量。須菩提。於意雲何。東方虛空可思量不。不也。世尊。須菩提。南西北方。四維上下。虛空可思量不。不也。世尊。須菩提。菩薩無住相布施。福德亦復如是。不可思量。須菩提。菩薩但應如所教住。
\chapter{第五品-如理實見分}
須菩提。於意雲何。可以身相見如來不。不也。世尊。不可以身相得見如來。何以故。如來所說身相。即非身相。佛告須菩提。凡所有相。皆是虛妄。若見諸相非相。即見如來。
\chapter{第六品-正信希有分}
須菩提白佛言。世尊。頗有眾生。得聞如是言說章句。生實信不。佛告須菩提。莫作是說。如來滅後。後五百歲。有持戒修福者。於此章句。能生信心。以此為實。當知是人。不於壹佛二佛三四五佛而種善根。已於無量千萬佛所種諸善根。聞是章句。乃至壹念生凈信者。須菩提。如來悉知悉見。是諸眾生。得如是無量福德。何以故。是諸眾生無復我相。人相。眾生相。壽者相。無法相。亦無非法相。何以故。是諸眾生。若心取相。即為著我人眾生壽者。若取法相。即著我人眾生壽者。何以故。若取非法相,即著我人眾生壽者。是故不應取法。不應取非法。以是義故。如來常說。汝等比丘。知我說法。如筏喻者。法尚應舍。何況非法。
\chapter{第七品-無得無說分}
須菩提。於意雲何。如來得阿耨多羅三藐三菩提耶。如來有所說法耶。須菩提言。如我解佛所說義。無有定法。名阿耨多羅三藐三菩提。亦無有定法。如來可說。何以故。如來所說法。皆不可取。不可說。非法非非法。所以者何。壹切賢聖,皆以無為法而有差別。
\chapter{第八品-依法出生分}
須菩提。於意雲何。若人滿三千大千世界七寶。以用布施。是人所得福德。寧為多不。須菩提言。甚多。世尊。何以故。是福德即非福德性。是故如來說福德多。若復有人。於此經中受持乃至四句偈等。為他人說。其福勝彼。何以故。須菩提。壹切諸佛。及諸佛阿耨多羅三藐三菩提法。皆從此經出。須菩提。所謂佛法者。即非佛法。
\chapter{第九品-壹相無相分}
須菩提。於意雲何。須陀洹能作是念。我得須陀洹果不。須菩提言。不也。世尊。何以故。須陀洹名為入流。而無所入。不入色聲香味觸法。是名須陀洹,須菩提。於意雲何。斯陀含能作是念。我得斯陀含果不。須菩提言。不也。世尊。何以故。斯陀含名壹往來。而實無往來。是名斯陀含。須菩提。於意雲何。阿那含能作是念。我得阿那含果不。須菩提言。不也。世尊。何以故。阿那含名為不來,而實無不來。是故名阿那含。須菩提。於意雲何。阿羅漢能作是念。我得阿羅漢道不。須菩提言。不也。世尊。何以故。實無有法名阿羅漢。世尊。若阿羅漢作是念。我得阿羅漢道。即為著我人眾生壽者。世尊。佛說我得無諍三昧。人中最為第壹。是第壹離欲阿羅漢。世尊。我不作是念。我是離欲阿羅漢。世尊。我若作是念。我得阿羅漢道。世尊則不說須菩提。是樂阿蘭那行者。以須菩提實無所行。而名須菩提。是樂阿蘭那行。
\chapter{第十品-莊嚴凈土分}
佛告須菩提。於意雲何。如來昔在然燈佛所。於法有所得不。不也。世尊。如來在然燈佛所。於法實無所得。須菩提。於意雲何。菩薩莊嚴佛土不。不也。世尊。何以故。莊嚴佛土者。即非莊嚴。是名莊嚴。是故須菩提。諸菩薩摩訶薩。應如是生清凈心。不應住色生心。不應住聲香味觸法生心。應無所住而生其心。須菩提。譬如有人。身如須彌山王,於意雲何。是身為大不。須菩提言。甚大。世尊。何以故。佛說非身。是名大身。

\chapter{第十壹品-無為福勝分}
須菩提。如恒河中所有沙數。如是沙等恒河。於意雲何。是諸恒河沙。寧為多不。須菩提言。甚多。世尊。但諸恒河尚多無數。何況其沙。須菩提。我今實言告汝。若有善男子。善女人。以七寶滿爾所恒河沙數三千大千世界。以用布施。得福多不。須菩提言。甚多。世尊。佛告須菩提。若善男子。善女人。於此經中。乃至受持四句偈等。為他人說。而此福德。勝前福德。
\chapter{第十二品-尊重正教分}
復次。須菩提。隨說是經。乃至四句偈等。當知此處。壹切世間天人阿修羅。皆應供養。如佛塔廟。何況有人。盡能受持讀誦。須菩提。當知是人。成就最上第壹希有之法。若是經典所在之處。即為有佛。若尊重弟子。
\chapter{第十三品-如法受持分}
爾時。須菩提白佛言。世尊。當何名此經。我等雲何奉持。佛告須菩提。是經名為金剛般若波羅蜜。以是名字。汝當奉持。所以者何。須菩提。佛說般若波羅蜜。即非般若波羅蜜。是名般若波羅蜜。須菩提。於意雲何。如來有所說法不。須菩提白佛言。世尊。如來無所說。須菩提。於意雲何。三千大千世界所有微塵。是為多不。須菩提言。甚多。世尊。須菩提。諸微塵。如來說非微塵。是名微塵。如來說世界。即非世界。是名世界。須菩提。於意雲何。可以三十二相見如來不。不也。世尊。不可以三十二相得見如來。何以故。如來說三十二相。即是非相。是名三十二相。須菩提。若有善男子。善女人。以恒河沙等身命布施。若復有人。於此經中。乃至受持四句偈等。為他人說。其福甚多。
\chapter{第十四品-離相寂滅分}
爾時須菩提。聞說是經。深解義趣。涕淚悲泣。而白佛言。希有世尊。佛說如是甚深經典。我從昔來所得慧眼。未曾得聞如是之經。世尊。若復有人得聞是經。信心清凈。則生實相。當知是人。成就第壹希有功德。世尊。是實相者。即是非相。是故如來說名實相。世尊。我今得聞如是經典。信解受持。不足為難。若當來世。後五百歲。其有眾生。得聞是經。信解受持。是人即為第壹希有。
何以故。此人無我相。無人相。無眾生相。無壽者相。所以者何。我相即是非相。人相眾生相壽者相即是非相。何以故。離壹切諸相。即名諸佛。佛告須菩提。如是如是。若復有人。得聞是經。不驚不怖不畏。當知是人甚為希有。何以故。須菩提。如來說第壹波羅蜜。即非第壹波羅蜜。是名第壹波羅蜜。須菩提。忍辱波羅蜜。如來說非忍辱波羅蜜。是名忍辱波羅蜜。何以故。須菩提!如我昔為歌利王割截身體。我於爾時。無我相。無人相。無眾生相。無壽者相。何以故。我於往昔節節支解時。若有我相人相眾生相壽者相。應生嗔恨。須菩提。又念過去於五百世作忍辱仙人。於爾所世。無我相。無人相。無眾生相。無壽者相。是故須菩提。菩薩應離壹切相。發阿耨多羅三藐三菩提心。不應住色生心。不應住聲香味觸法生心。應生無所住心。若心有住即為非住,是故佛說菩薩心不應住色布施。須菩提。菩薩為利益壹切眾生故。應如是布施。如來說壹切諸相。即是非相。又說壹切眾生。即非眾生。須菩提。如來是真語者。實語者。如語者。不誑語者。不異語者。須菩提。如來所得法。此法無實無虛。須菩提。若菩薩心。住於法而行布施。如人入暗,即無所見。若菩薩心不住法而行布施。如人有目。日光明照。見種種色。須菩提。當來之世。若有善男子。善女人。能於此經受持讀誦。即為如來。以佛智慧。悉知是人。悉見是人。皆得成就無量無邊功德。
\chapter{第十五品-持經功德分}
須菩提。若有善男子。善女人。初日分。以恒河沙等身布施。中日分。復以恒河沙等身布施。後日分。亦以恒河沙等身布施。如是無量百千萬億劫。以身布施。若復有人,聞此經典。信心不逆。其福勝彼。何況書寫受持讀誦。為人解說。須菩提。以要言之。是經有不可思議。不可稱量。無邊功德。如來為發大乘者說。為發最上乘者說。若有人能受持讀誦。廣為人說。如來悉知是人。悉見是人。皆得成就不可量。不可稱。無有邊。不可思議功德。如是人等。即為荷擔如來阿耨多羅三藐三菩提。何以故。須菩提。若樂小法者。著我見人見眾生見壽者見。即於此經。不能聽受讀誦。為人解說。須菩提。在在處處。若有此經。壹切世間天人阿修羅。所應供養。當知此處。即為是塔。皆應恭敬。作禮圍繞。以諸華香而散其處。
\chapter{第十六品-能凈業障分}
復次。須菩提。若善男子。善女人。受持讀誦此經。若為人輕賤。是人先世罪業。應墮惡道。以今世人輕賤故。先世罪業即為消滅。當得阿耨多羅三藐三菩提。須菩提。我念過去無量阿僧祗劫。於然燈佛前。得值八百四千萬億那由他諸佛。悉皆供養承事。無空過者。若復有人。於後末世。能受持讀誦此經。所得功德。於我所供養諸佛功德。百分不及壹。千萬億分乃至算數譬喻所不能及。須菩提。若善男子。善女人。於後末世。有受持讀誦此經。所得功德。我若具說者。或有人聞。心即狂亂。狐疑不信。須菩提。當知是經義不可思議。果報亦不可思議。
\chapter{第十七品-究竟無我分}
爾時須菩提白佛言。世尊。善男子。善女人。發阿耨多羅三藐三菩提心。雲何應住?雲何降伏其心?佛告須菩提。善男子。善女人。發阿耨多羅三藐三菩提心者。當生如是心。我應滅度壹切眾生。滅度壹切眾生已。而無有壹眾生實滅度者。何以故。須菩提。若菩薩有我相人相眾生相壽者相,即非菩薩。所以者何。須菩提。實無有法發阿耨多羅三藐三菩提心者。須菩提。於意雲何。如來於然燈佛所。有法得阿耨多羅三藐三菩提不。不也。世尊。如我解佛所說義。佛於然燈佛所。無有法得阿耨多羅三藐三菩提。佛言。如是如是。須菩提。實無有法如來得阿耨多羅三藐三菩提。須菩提。若有法如來得阿耨多羅三藐三菩提者。然燈佛則不與我授記。汝於來世。當得作佛。號釋迦牟尼。以實無有法得阿耨多羅三藐三菩提。是故然燈佛與我授記。作是言。汝於來世。當得作佛。號釋迦牟尼。何以故。如來者。即諸法如義。若有人言。如來得阿耨多羅三藐三菩提。須菩提。實無有法。佛得阿耨多羅三藐三菩提。須菩提。如來所得阿耨多羅三藐三菩提。於是中無實無虛。是故如來說壹切法皆是佛法。須菩提。所言壹切法者。即非壹切法。是故名壹切法。須菩提。譬如人身長大。須菩提言。世尊。如來說人身長大。即為非大身。是名大身。須菩提。菩薩亦如是。若作是言。我當滅度無量眾生。即不名菩薩。何以故。須菩提。實無有法名為菩薩。是故佛說。壹切法無我無人無眾生無壽者。須菩提。若菩薩作是言。我當莊嚴佛土。是不名菩薩。何以故。如來說莊嚴佛土者。即非莊嚴。是名莊嚴。須菩提。若菩薩通達無我法者。如來說名真是菩薩。
\chapter{第十八品-壹體同觀分}
須菩提。於意雲何。如來有肉眼不。如是。世尊。如來有肉眼。須菩提。於意雲何。如來有天眼不。如是。世尊。如來有天眼。須菩提。於意雲何。如來有慧眼不。如是。世尊。如來有慧眼。須菩提。於意雲何。如來有法眼不。如是。世尊。如來有法眼。須菩提。於意雲何。如來有佛眼不。如是。世尊。如來有佛眼。須菩提。於意雲何。如恒河中所有沙。佛說是沙不。如是。世尊。如來說是沙。須菩提。於意雲何。如壹恒河中所有沙。有如是沙等恒河。是諸恒河所有沙數佛世界,如是寧為多不。甚多。世尊。佛告須菩提。爾所國土中。所有眾生,若幹種心。如來悉知。何以故。如來說諸心皆為非心。是名為心。所以者何。須菩提。過去心不可得。現在心不可得。未來心不可得。
\chapter{第十九品-法界通化分}
須菩提。於意雲何。若有人滿三千大千世界七寶。以用布施。是人以是因緣。得福多不。如是。世尊。此人以是因緣。得福甚多。須菩提。若福德有實。如來不說得福德多。以福德無故。如來說得福德多。
\chapter{第二十品-離色離相分}
須菩提。於意雲何。佛可以具足色身見不。不也。世尊。如來不應以具足色身見。何以故。如來說。具足色身。即非具足色身。是名具足色身。須菩提。於意雲何。如來可以具足諸相見不。不也。世尊。如來不應以具足諸相見。何以故。如來說諸相具足。即非具足。是名諸相具足。
\chapter{第二十壹品-非說所說分}
須菩提。汝勿謂如來作是念。我當有所說法。莫作是念。何以故。若人言如來有所說法。即為謗佛。不能解我所說故。須菩提。說法者。無法可說。是名說法。爾時慧命須菩提白佛言。世尊。頗有眾生。於未來世。聞說是法。生信心不。佛言。須菩提。彼非眾生。非不眾生。何以故。須菩提。眾生眾生者。如來說非眾生。是名眾生。
\chapter{第二十二品-無法可得分}
須菩提白佛言。世尊。佛得阿耨多羅三藐三菩提。為無所得耶。佛言。如是。如是。須菩提。我於阿耨多羅三藐三菩提。乃至無有少法可得。是名阿耨多羅三藐三菩提。

\chapter{第二十三品-凈心行善分}
復次。須菩提。是法平等。無有高下。是名阿耨多羅三藐三菩提。以無我無人無眾生無壽者。修壹切善法。即得阿耨多羅三藐三菩提。須菩提。所言善法者。如來說即非善法。是名善法。
\chapter{第二十四品-福智無比分}
須菩提。若三千大千世界中。所有諸須彌山王。如是等七寶聚。有人持用布施。若人以此般若波羅蜜經。乃至四句偈等。受持讀誦。為他人說。於前福德。百分不及壹。百千萬億分。乃至算數譬喻所不能及。
\chapter{第二十五品-化無所化分}
須菩提。於意雲何。汝等勿謂如來作是念。我當度眾生。須菩提。莫作是念。何以故。實無有眾生如來度者。若有眾生如來度者。如來即有我人眾生壽者。須菩提。如來說有我者。即非有我。而凡夫之人以為有我。須菩提。凡夫者。如來說即非凡夫。是名凡夫。
\chapter{第二十六品-法身非相分}
須菩提。於意雲何。可以三十二相觀如來不。須菩提言。如是如是以三十二相觀如來。佛言。須菩提。若以三十二相觀如來者。轉輪聖王即是如來。須菩提白佛言。世尊。如我解佛所說義。不應以三十二相觀如來。爾時。世尊而說偈言。若以色見我。以音聲求我。是人行邪道。不能見如來。
\chapter{第二十七品-無斷無滅分}
須菩提。汝若作是念。如來不以具足相故。得阿耨多羅三藐三菩提。須菩提。莫作是念。如來不以具足相故。得阿耨多羅三藐三菩提。須菩提。汝若作是念。發阿耨多羅三藐三菩提心者。說諸法斷滅。莫作是念。何以故。發阿耨多羅三藐三菩提心者。於法不說斷滅相。
\chapter{第二十八品-不受不貪分}
須菩提。若菩薩以滿恒河沙等世界七寶。持用布施。若復有人知壹切法無我。得成於忍。此菩薩勝前菩薩所得功德。何以故。須菩提。以諸菩薩不受福德故。須菩提白佛言。世尊。雲何菩薩不受福德。須菩提。菩薩所作福德。不應貪著。是故說不受福德。
\chapter{第二十九品-威儀寂凈分}
須菩提。若有人言。如來若來若去。若坐若臥。是人不解我所說義。何以故。如來者。無所從來。亦無所去。故名如來。
\chapter{第三十品-壹合理相分}
須菩提。若善男子。善女人。以三千大千世界碎為微塵。於意雲何。是微塵眾寧為多不。須菩提言。甚多。世尊。何以故。若是微塵眾實有者。佛即不說是微塵眾。所以者何。佛說。微塵眾。即非微塵眾。是名微塵眾。世尊。如來所說三千大千世界。即非世界。是名世界。何以故。若世界實有者。即是壹合相。如來說。壹合相。即非壹合相。是名壹合相。須菩提。壹合相者。即是不可說。但凡夫之人貪著其事。

\chapter{第三十壹品-知見不生分}
須菩提。若人言。佛說我見人見眾生見壽者見。須菩提。於意雲何。是人解我所說義不。不也。世尊。是人不解如來所說義。何以故。世尊說。我見人見眾生見壽者見,即非我見人見眾生見壽者見,是名我見人見眾生見壽者見。須菩提。發阿耨多羅三藐三菩提心者。於壹切法。應如是知。如是見。如是信解。不生法相。須菩提。所言法相者。如來說即非法相。是名法相。
\chapter{第三十二品-應化非真分}
須菩提。若有人以滿無量阿僧祗世界七寶持用布施。若有善男子。善女人發菩提心者。持於此經。乃至四句偈等。受持讀誦。為人演說。其福勝彼。雲何為人演說。不取於相。如如不動。何以故。壹切有為法。如夢幻泡影。如露亦如電。應作如是觀。佛說是經已。長老須菩提。及諸比丘。比丘尼。優婆塞。優婆夷。壹切世間天人阿修羅。聞佛所說。皆大歡喜。信受奉行。

\end{document}
